\section{Sensor Data Units}\label{section:parameterUnits}

The outputs of sensors need to be processed to meaningful values.
The Table \ref{table:rain_data} shows the unit for the sensor values.
`Raw Units' in the table means the unit for the packtized data, which you can get directly from the coresense boards, and `Processed Units' means the unit which can be used after data conversion.


\subsection{Rain Gauge}
The conversion method for the rain gauge is provided as comments in the table.\\


\begin{table}[H]
    \centering {
    \begin{tabular}{|c|c|c|c|c|}
        \hline
        \rowcolor{black!8}
        \textbf{Sensor} & \textbf{Sensor ID} & \textbf{Raw Units} & \textbf{Processed Units} & \textbf{Comments} \\ 
        \hline
        \multirow{2}{*}{Rain Gauge} & \multirow{2}{*}{0xFC} & \multirow{2}{*}{integer} & \multirow{2}{*}{in. or mm} & One pendant event = 0.01 in. = 0.254 mm \\
        & & & & precipitation = output $\times$ 0.01 (or 0.254) \\ 
        \hline
    \end{tabular}
    }
    \caption{Sub-packet data for sensors}
    \label{table:rain_data}
\end{table}


\subsection{Soil Moisture Sensor}
And data from the Metsense board of this soil sensor does not need a conversion method since the data from the board are the processed values.


\begin{table}[H]
    \centering {
    \begin{tabular}{|c|c|c|c|c|}
        \hline
        \rowcolor{black!8}
        \textbf{Sensor} & \textbf{Sensor ID} & \textbf{Raw Units} & \textbf{Processed Units} & \textbf{Comments} \\ 
        \hline
        Soil sensor & 0xFC & No unit, dS/m, $\degree$C & No unit, dS/m, $\degree$C & \\ \hline
    \end{tabular}
    }
    \caption{Sub-packet for rain gauge}
    \label{table:soil_data}
\end{table}

