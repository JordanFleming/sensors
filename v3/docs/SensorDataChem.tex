
\subsubsection{Chemsense:}

\paragraph{$\bullet$ Chemical sensors -- Total reducing gases, Nitrogen dioxide, Ozone, Hydrogen sulphide, Total oxidizing gases, Carbon monoxide, and Sulfur dioxide:}
These parameters provide the current output of the electrochemical ToR, NO$_2$, O$_3$, H$_2$S, ToX, CO, and SO$_2$ sensor.
Each of the cell current is quantified using an AFE that uses a
24-bit ADC to convert it into a signed digital value. This value is
represented in Format 5.

\begin{table}[h!]
    \centering
    \caption{Sub-packet of a temperature sensor, TMP421}
    \begin{tabular}{|c|c|c|}
        \hline
        \rowcolor{black!8}
        \textbf{Sensor ID} (0x15 $\sim$ 0x1C) & \textbf{0x83} & \textbf{Raw concentartion in Format 5}\\
        \hline
        Byte[0] & Byte[1] & Bytes[2 -- 4] \\ \hline
    \end{tabular}
\end{table}


\paragraph{$\bullet$ SHT25:}

SHT25 is a temperature and relative humidity sensor. 100th of temperature and relative humidity are
encoded in Format 2.

\begin{table}[h!]
    \centering
    \caption{Sub-packet of a temperature and relative humidity sensor, SHT25}
    \begin{tabular}{|c|c|c|c|}
        \hline
        \rowcolor{black!8}
        \textbf{0x1D} & \textbf{0x84} & \textbf{100th of Temperature in Format 2} & \textbf{100th of Relative humidity in Format 2}\\
        \hline
        Byte[0] & Byte[1] & Bytes[2 -- 4] & Bytes[5 -- 6] \\ \hline
    \end{tabular}
\end{table}



\paragraph{$\bullet$ LPS25H:}

LPS25H is a temperature and barometric pressure sensor. 100th of temperature and barometric pressureare
encoded in Format 2 and Format 4 respectively.

\begin{table}[h!]
    \centering
    \caption{Sub-packet of a temperature and barometric pressure sensor, LPS25H}
    \begin{tabular}{|c|c|c|c|}
        \hline
        \rowcolor{black!8}
        \textbf{0x1E} & \textbf{0x85} & \textbf{100th of Temperature in Format 2} & \textbf{barometric pressure in Format 4}\\
        \hline
        Byte[0] & Byte[1] & Bytes[2 -- 4] & Bytes[5 -- 6] \\ \hline
    \end{tabular}
\end{table}


\paragraph{$\bullet$ Si1145:}
Si1145 is a light sensor for three factors; ultra-violet, visible, and infrared. 
The raw output from the sensor is encoded Format 1.

\begin{table}[h!]
    \centering
    \caption{Sub-packet of an ultra-violet sensor, Si1145}
    \begin{tabular}{|c|c|c|c|c|}
        \hline
        \rowcolor{black!8}
        \textbf{0x1F} & \textbf{0x86} & \textbf{Raw UV in Format 1} & \textbf{Raw visible light in Format 1} & \textbf{Raw IR in Format 1}\\
        \hline
        Byte[0] & Byte[1] & Bytes[2 -- 4] & Bytes[5 -- 6] & Bytes[7 -- 8] \\ \hline
    \end{tabular}
\end{table}


\paragraph{$\bullet$ Chemsense MAC address:}

This is a six byte ID that uniquely identifies each Chemsense board. The Unique ID uses Format 3
for encoding and the arrangement is listed below.


\begin{table}[h!]
    \centering
    \caption{Sub-packet of Chemsense MAC address}
    \begin{tabular}{|c|c|c|c|c|c|c|c|}
        \hline
        \rowcolor{black!8}
        \textbf{0x20} & \textbf{0x86} & \textbf{Chemsense board MAC address in Format 3}\\
        \hline
        Byte[0] & Byte[1] & Bytes[2 -- 7] \\ \hline
    \end{tabular}
\end{table}


\paragraph{$\bullet$ ADC Temperatures -- CO ADC Temp, IAQ/IRR ADC Temp, O3/NO2 ADC Temp, SO2/H2S ADC Temp, and CO LMT Temp:}
Chemsense board measures temperature of sensor ADCs. This includes five parameters and 
all of them give ADC temperature in 100ths of degree celsious. Format 2 is used for encoding and the arrangement is listed below. 


\begin{table}[h!]
    \centering
    \caption{Sub-packet of ADC Temperatures}
    \begin{tabular}{|c|c|c|}
        \hline
        \rowcolor{black!8}
        \textbf{Sensor ID} (0x21 $\sim$ 0x25) & \textbf{0x82} & \textbf{Temperature in Format 2}\\
        \hline
        Byte[0] & Byte[1] & Bytes[2 -- 3] \\ \hline
    \end{tabular}
\end{table}


\paragraph{$\bullet$ Accelerometer:}
The accelerations in three orthogonal directions, x, y and z, as a multiple of acceleration are obtained from the sensor, and a vibration index is calculated. 
Acceleration data are encoded in Format 2, and vibration index is encoded in Format 4. 

\begin{table}[h!]
    \centering
    \caption{Sub-packet of Accelerometer}
    \begin{tabular}{|c|c|c|c|c|c|}
        \hline
        \rowcolor{black!8}
        \textbf{0x26} & \textbf{0x89} & \textbf{A(x) in Format 2} & \textbf{A(y) in Format 2} & \textbf{A(z) in Format 2}& \textbf{Vib. in Format 4}\\
        \hline
        Byte[0] & Byte[1] & Bytes[2 -- 3] & Bytes[4 -- 5] & Bytes[6 -- 7] & Bytes[8 -- 10] \\ \hline
    \end{tabular}
\end{table}


\paragraph{$\bullet$ Gyro:}
The gyro in three orthogonal directions, x, y and z, as a multiple of acceleration are obtained from the sensor, and a orientation index is calculated. 
Gyro data are encoded in Format 2, and orientation index is encoded in Format 4.

\begin{table}[h!]
    \centering
    \caption{Sub-packet of Gyro}
    \begin{tabular}{|c|c|c|c|c|c|}
        \hline
        \rowcolor{black!8}
        \textbf{0x27} & \textbf{0x89} & \textbf{O(x) in Format 2} & \textbf{O(y) in Format 2} & \textbf{O(z) in Format 2}& \textbf{Index in Format 4}\\
        \hline
        Byte[0] & Byte[1] & Bytes[2 -- 3] & Bytes[4 -- 5] & Bytes[6 -- 7] & Bytes[8 -- 10] \\ \hline
    \end{tabular}
\end{table}


