\subsubsection{Metsense}

\paragraph{$\bullet$ Metsense/Lightsense MAC address: }

This is a six byte ID that uniquely identifies each Airsense board. This MAC address is also applied to each Lightsense board which has the same board number. The ID is provided by a DS2401 1-Wire DSN chip. The 1-byte family ID and CRC provided by the DSN chip are omitted, and the rest six bytes are used as the Unique ID. The Unique ID uses Format 3 for encoding and the arrangement is listed below.


\begin{table}[h!]
    \centering
    \caption{Sub-packet of Met/lightsense board MAC address}
    \begin{tabular}{|c|c|c|}
        \hline
        \rowcolor{black!8}
        \textbf{0x00} & \textbf{0x86} & \textbf{MAC address in Format 3} \\
        \hline
        Byte[0] & Byte[1] & Bytes[2 -- 7]\\ \hline
    \end{tabular}
\end{table}
\par

\paragraph{$\bullet$ TMP112, TSYS01:}

TMP112 and TSYS01 are digital temperature sensors, which provides the temperature values
in centigrade.

\begin{table}[h!]
    \centering
    \caption{Sub-packet of a temperature sensor, TMP112}
    \begin{tabular}{|c|c|c|}
        \hline
        \rowcolor{black!8}
        \textbf{Sensor ID} (0x01 or 0x09) & \textbf{0x82} & \textbf{Temperature in Format 6} \\
        \hline
        Byte[0] & Byte[1] & Bytes[2 -- 3]\\ \hline
    \end{tabular}
\end{table}


\paragraph{$\bullet$ HTU21D:}
HTU21D is a digital temperature and relative humidity sensor, which provides
relative humidity value (\%RH) and temperature value in centigrade.
\\

\begin{table}[h!]
    \centering
    \caption{Sub-packet of a temperature and relative humidity sensor, HTU21D}
    \begin{tabular}{|c|c|c|c|}
        \hline
        \rowcolor{black!8}
        \textbf{0x02} & \textbf{0x84} & \textbf{Temperature in Format 6} & \textbf{Relative humidity in Format 6}\\
        \hline
        Byte[0] & Byte[1] & Bytes[2 -- 3] & Bytes[4 -- 5] \\ \hline
    \end{tabular}
\end{table}


\paragraph{$\bullet$ BMP180:}

BMP180 is an digital temperature and barometric pressure sensor,
which provides temperature in centigrade and pressure in hectopascals.
\\


\begin{table}[h!]
    \centering
    \caption{Sub-packet of a temperature and barometric pressure sensor, BMP180}
    \begin{tabular}{|c|c|c|c|}
        \hline
        \rowcolor{black!8}
        \textbf{0x04} & \textbf{0x84} & \textbf{Temperature in Format 6} & \textbf{Barometric pressure in Format 4}\\
        \hline
        Byte[0] & Byte[1] & Bytes[2 -- 3] & Bytes[4 -- 5] \\ \hline
    \end{tabular}
\end{table}


\paragraph{$\bullet$ PR103J2:}

PR103J2 is an analog temperature sensor whose resistance changes with change in temperature.
The sensor is implemented in a voltage divider circuit, and the center-tap voltage is converted and packed into Format 1 using a 10-bit ADC.


\begin{table}[h!]
    \centering
    \caption{Sub-packet of a temperature sensor, PR103J2}
    \begin{tabular}{|c|c|c|}
        \hline
        \rowcolor{black!8}
        \textbf{0x05} & \textbf{0x82} & \textbf{Voltage output in Format 1} \\
        \hline
        Byte[0] & Byte[1] & Bytes[2 -- 3]\\ \hline
    \end{tabular}
\end{table}


\paragraph{$\bullet$ TSL250RD:}

TSL250RD is an analog visible light sensor that produces an analog voltage that is
representative of the irradiance measured in $\mu$W/cm$^2$. The output voltage of the sensor
is converted and packed into Format 1 using a 10-bit ADC.


\begin{table}[h!]
    \centering
    \caption{Sub-packet of a light intensity sensor, TSL250}
    \begin{tabular}{|c|c|c|}
        \hline
        \rowcolor{black!8}
        \textbf{0x06} & \textbf{0x82} & \textbf{Voltage output in Format 1} \\
        \hline
        Byte[0] & Byte[1] & Bytes[2 -- 3]\\ \hline
    \end{tabular}
\end{table}


\paragraph{$\bullet$ MMA8452Q:}

MMA8452Q is a digital three-axis accelerometer. The accelerations in three orthogonal directions,
x, y and z, as a multiple of acceleration due to gravity (g) are obtained from the sensor,
and a vibration value (represented as multiple of g) is calculated using high-frequency
time series data from the three axis.

\begin{table}[h!]
    \centering
    \caption{Sub-packet of a three-axis accelerometer, MMA8452Q}
    \begin{tabular}{|c|c|c|c|c|c|}
        \hline
        \rowcolor{black!8}
        \textbf{0x07} & \textbf{0x88} & Acc(x) in Format 6 & Acc(y) in Format 6 & Acc(z) in Format 6 & Vib. in Format 6\\
        \hline
        Byte[0] & Byte[1] & Bytes[2 -- 3] & Bytes[4 -- 5] & Bytes[6 -- 7] & Bytes[8 -- 9] \\ \hline
    \end{tabular}
\end{table}



\paragraph{$\bullet$ SPV1840LR5H-B:}

SPV1840LR5H is a MEMS microphone that is sampled at high frequency to obtain
the peaks and calculate the sound intensity for a time window. The raw calculated
intensity is represented as a 16-bit integer value using Format 1.
\\

\begin{table}[h!]
    \centering
    \caption{Sub-packet of a sound level sensor, SPV1840LR5H-B}
    \begin{tabular}{|c|c|c|}
        \hline
        \rowcolor{black!8}
        \textbf{0x08} & \textbf{0x82} & \textbf{Voltage output in Format 1} \\
        \hline
        Byte[0] & Byte[1] & Bytes[2 -- 3]\\ \hline
    \end{tabular}
\end{table}


\subsubsection{Lightsense}

\paragraph{$\bullet$ HMC5883L:}
HMC5883L is a digital three-axis magnetometer. The magnetic field strengths in three orthogonal directions,
x, y and z are obtained from the sensor.
\\

\begin{table}[h!]
    \centering
    \caption{Sub-packet of a three-axis magnetometer, HMC5883L}
    \begin{tabular}{|c|c|c|c|c|}
        \hline
        \rowcolor{black!8}
        \textbf{0x0A} & \textbf{0x86} & \textbf{Strength Hx in Format 8} & \textbf{Strength Hy in Format 8} & \textbf{Strength Hz in Format 8}\\
        \hline
        Byte[0] & Byte[1] & Bytes[2 -- 3] & Bytes[4 -- 5] & Bytes[6 -- 7] \\ \hline
    \end{tabular}
\end{table}



\paragraph{$\bullet$ HIH6130:}

HIH6130 is a digital temperature and relative humidity sensor, which provides
relative humidity value (\%RH) and temperature value in centigrade.
\\

\begin{table}[h!]
    \centering
    \caption{Sub-packet of a temperature and relative humidity sensor, HIH6130}
    \begin{tabular}{|c|c|c|c|}
        \hline
        \rowcolor{black!8}
        \textbf{0x0B} & \textbf{0x84} & \textbf{Temperature in Format 6} & \textbf{Relative Humidity in Format 6}\\
        \hline
        Byte[0] & Byte[1] & Bytes[2 -- 3] & Bytes[4 -- 5] \\ \hline
    \end{tabular}
\end{table}


\paragraph{$\bullet$ APDS-9006-020, TSL260, TSL250, MLX75305, and ML8511:}

APDS-9006-020, TSL260, TSL250, MLX75305, and ML8511 are analog light intensity sensors that produce the analog voltage 
representing the general luminance or the irradiance measured in $\mu$W/cm$^2$. The output voltage of the sensor
is converted and packed into Format 1 using a 16-bit ADC.

\begin{table}[h!]
    \centering
    \caption{Sub-packet of light intensity sensors, APDS-9006-020, TSL260, TSL250, MLX75305, and ML8511}
    \begin{tabular}{|c|c|c|c|}
        \hline
        \rowcolor{black!8}
        \textbf{Sensor ID} (0x0C $\sim$ 0x10) & \textbf{0x82} & \textbf{Voltage output in Format 1}\\
        \hline
        Byte[0] & Byte[1] & Bytes[2 -- 3] \\ \hline
    \end{tabular}
\end{table}


\paragraph{$\bullet$ TMP421:}
TMP421 is a digital temperature sensor, which provides the temperature values
in centigrade.

\begin{table}[h!]
    \centering
    \caption{Sub-packet of a temperature sensor, TMP421}
    \begin{tabular}{|c|c|c|}
        \hline
        \rowcolor{black!8}
        \textbf{0x13} & \textbf{0x82} & \textbf{Temperature in Format 6}\\
        \hline
        Byte[0] & Byte[1] & Bytes[2 -- 3] \\ \hline
    \end{tabular}
\end{table}


\paragraph{$\bullet$ SPV1840LR5H-B}

SPV1840LR5H is a MEMS microphone that is sampled at high frequency to obtain
the peaks and calculate the sound intensity for a time window. The raw calculated
intensity is represented as a 16-bit integer value using Format 1.

\begin{table}[h!]
    \centering
    \caption{Sub-packet of a sound level sensor, SPV1840LR5H-B}
    \begin{tabular}{|c|c|c|}
        \hline
        \rowcolor{black!8}
        \textbf{0x14} & \textbf{0x82} & \textbf{Voltage output in Format 1}\\
        \hline
        Byte[0] & Byte[1] & Bytes[2 -- 3] \\ \hline
    \end{tabular}
\end{table}

