\section{Data Transmission} \label{section:overall}

The data from the sensor boards is sent as a formatted unit of data --- a transmission
packet. A transmission packet is composed of several data sub-packets, each
of which carries information pertaining to the parameter listed in the sub-packet.
The transmission packet format and the data sub-packets are described here.

\subsection{Transmission Packet}
A transmission packet can be broken down into 6 segments.
The structure of the transmission packet (and the data sub-packet) relies on
byte positions and predefined values for some segments. The first segment
is the start byte or
the preamble. The preamble is followed by the packet sequence number and protocol
version, each of which are 4 bits long and are together packed into a single byte.
Next, a one byte field that reports the length of the data which follows it
immediately. The data segment is
followed by a single CRC byte, and finally the packet ends with a one byte
postscript. The table below lists the packet and the static values, if any,
for each of the segments.

\begin{table}[H]
    \centering
    {\rowcolors{2}{black!8}{black!2}
    \begin{tabular}{|c|c|c|c|}
        \hline
        \textbf{Field} & \textbf{Value} & \textbf{Segment} & \textbf{Length}\\
        \hline
        \hline
        Preamble & 0xAA & 1 & 1 Byte\\
        Packet Sequence Number & Variable & 2 & 1 Nibble\\
        Protocol version & 0x00 & 2 & 1 Nibble\\
        Length of data (not whole packet) & 0x9D & 3 & 1 Byte\\
        Data & \dots & 4 & 157 Bytes\\
        CRC of data (not whole packet) & Variable & 5 & 1 Byte\\
        Postscript & 0x55 & 6 & 1 Byte\\
        \hline
    \end{tabular}
    }
    \caption{Transmission Packet Segments}
    \label{table:packsegments}
\end{table}

The table below illustrates how the segments are organized in a transmission
packet.

\begin{table}[H]
\centering
\begin{tabular}{|c|c|c|c|c|c|}
\hline
% column1a & column2a \\
\noalign{\hrule height 2pt}
\multicolumn{1}{!{\vrule width 2pt}c!{\vrule width 1pt}}{Preamble} &
\multicolumn{1}{!{\vrule width 2pt}c!{\vrule width 1pt}}{Seq. Number | Prot. Ver.} &
\multicolumn{1}{!{\vrule width 2pt}c!{\vrule width 1pt}}{Data Length} &
\multicolumn{1}{!{\vrule width 2pt}c!{\vrule width 1pt}}{Data} &
\multicolumn{1}{!{\vrule width 2pt}c!{\vrule width 1pt}}{CRC} &
\multicolumn{1}{!{\vrule width 2pt}c!{\vrule width 1pt}}{Postscript}\\
\noalign{\hrule height 2pt}
Byte 0 & Byte 1 & Byte 2 & up to 255 Bytes & Penultimate Byte & Final Byte \\
\hline
\end{tabular}
\caption{Transmission Packet structure}
    \label{table:packstruct}
\end{table}

\subsection{Data Packer CRC} \label{ssec:crc-calc}

To validate the data transmitted from the sensor board, a CRC value for the data is
calculated and transmitted as part of the data packet. The Maxim 1-Wire
CRC polynomial is used for calculating the CRC.  On receiving the data packet, the CRC
of the data packet is recalculated and compared with the value transmitted as part of
the packet. If the two CRC values match, the transmission is error-free.
The equivalent polynomial function of the CRC is:
\newline
\newline
$CRC = x^8 + x^5 + x^4 + 1. $
\newline
\newline
Further description of the Maxim 1-Wire CRC is available in Maxim Application Note 27. Below are
the Python and C implementations of the CRC calculator. The CRC implementations below take a
data byte and the previous CRC as inputs, and return the new CRC as return value.
\newline
\newline
\textbf{Python Code:}
\begin{mdframed}
\begin{lstlisting}
def calc_crc (data_byte,CRC_Value)
    CRC_Value = ord(data_byte) ^ CRC_Value
    for j in range(8):
    if (CRC_Value  & 0x01):
        CRC_Value  = (CRC_Value  >> 0x01) ^ 0x8C
    else:
        CRC_Value  =  CRC_Value  >> 0x01
return CRC_Value
\end{lstlisting}
\end{mdframed}

\vskip 0.1in
\textbf{C Code:}
\begin{mdframed}
\begin{lstlisting}
unsigned char  CRC_CALC (unsigned char data, unsigned char crc)
{
        unsigned char i;
        crc ^= data;
        for (i=0x00; i < 0x08; i++)
        {
                if (crc & 0x01)
                        crc = (crc >> 0x01) ^ 0x8C;
                else
                        crc =  crc >> 0x01;
        }

        return(crc);
}
\end{lstlisting}
\end{mdframed}


\subsection{Data Sub-packets} \label{ssec:sub-pack}

The data segment of the transmission packet is further broken down into many
sub-packets. The sub-packet starts with a source identifier. A one bit
validity field and seven bits ``length of the sub-packet'' field
are packed together as the next byte. The length field counts the number of
bytes following it which make up the sub-packet. The table below shows the organization
of a sub-packet. The validity bit is set to 0 if the sensor represented in the sub-packet
is dead, disabled, unconnected, unresponsive or if data could not be collected
from the sensor in the time window. Instead of recreating the full packet in
the buffer, sans the particular sub-packet, the sensor segment is marked as invalid.
If the field is set to 1, it indicates a valid measurement/reading. The seven bits length
field restricts the size of the sub-packet to 127 bytes.




\begin{table}[H]
\centering
\begin{tabular}{|c|c|c|}
\hline
% column1a & column2a \\
\noalign{\hrule height 2pt}
\multicolumn{1}{!{\vrule width 2pt}c!{\vrule width 1pt}}{Source ID} &
\multicolumn{1}{!{\vrule width 2pt}c!{\vrule width 1pt}}{1-bit Validity [0: invalid, 1: valid]| 7-bits Length} &
\multicolumn{1}{!{\vrule width 2pt}c!{\vrule width 1pt}}{Data} \\
\noalign{\hrule height 2pt}
Byte 0 & Byte 1 & up to 127 Bytes \\
\hline
\end{tabular}
\end{table}




%%%%%%%%%%%%%%%%%%%%
